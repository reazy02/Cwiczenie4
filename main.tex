\documentclass[aspectratio=169]{beamer}

\usetheme{NewPwr}

\usepackage[polish]{babel}
\usepackage[utf8]{inputenc}
\usepackage[T1]{fontenc}
\usepackage[export]{adjustbox}
\usepackage{graphicx}
\usepackage{carlito}

\usefonttheme{professionalfonts}

\title[Aerodynamika w F1]{Aerodynamika w F1}
\author{Bartłomiej Posłuszny}
\institute{}
\date{}

\begin{document}

%-------------------------------------------------
\begin{frame}
  \titlepage
  \vspace{0.1cm}
  \centering
  \includegraphics[
    width=0.65\textwidth,
    frame,
    margin=2pt
  ]{images/obrazek1.png}

\end{frame}

%-------------------------------------------------
\begin{frame}{Zaliczenie zadania}
    https://github.com/reazy02/Cwiczenie4
\end{frame}




%-------------------------------------------------
\begin{frame}{Spis treści}
  \tableofcontents
\end{frame}

%-------------------------------------------------
\section{Wprowadzenie}

\begin{frame}{Historia Formuły 1}
F1 to międzynarodowy wyścig samochodowy który trwa nieprzerwanie od początków lat 50-tych. Pierwszy wyścig odbył się na starym lotnisku RAF
oebcnie znanym jako tor Silverstone. Pierwsze samochody F1 posiadały silniki z przodu, polegając głównie na mocy silnika a ich aerodynamika opierała się głównie na kształcie podwozia.

Prawdziwy przełom nastąpił w latach 60 gdzie takie zespoły jak Lotus zaczeły eksperymentować z skrzydłami oraz lotkami (np.Lotus 49B) który generowały docisk aerodynamiczny wciskający bolid w tor.

\end{frame}
%------------------------------------------------
\begin{frame}
    \centering
    \includegraphics[width=0.9\paperwidth,height=0.9\paperheight,frame, margin=1pt,keepaspectratio]{images/obrazek2.jpg}
\end{frame}

%------------------------------------------------
\begin{frame}{Wstęp do aerodynamiki}
Zadaniem aerodynamiki jest wytworzenie siły dociskowej w celu przyciśnięcia opon bolidu do toru i zwiększenie przyczepności na zakrętach z jednoczesnym zminimalizowaniem oporu powietrza które zostaje wywołane przez turbulencję co spowolnia bolid.

  \centering
  \includegraphics[
    width=0.6\paperwidth,
    height=0.5\paperheight,
    frame,
    margin=1pt,
    keepaspectratio,
  ]{images/obrazek3.png}

\end{frame}



%------------------------------------------------
\begin{frame}{Znaczenie aerodynamiki}
Elementy aerodynamiczne pozwalają na minimalizowanie oporu powietrza ale i nie tylko, wpływają one także na wiele innych aspektów takich jak: 
\begin{itemize}
  \item Większą efektywność paliwową (mniejsze spalanie)
  \item Osiągi pojazdów, większa prędkość oraz przyspieszenie
  \item Doskonała przyczepność i lepsza kontrola co prowadzi do znacznie szybszego pokonywania zakrętów.
\end{itemize}
\end{frame}

%-------------------------------------------------
\section{Przyczepność i tarcie}

\begin{frame}{Źródła przyczepności}
Źródłem przyczepności samochodu F1 jest siła tarcia działająca między oponą a podłożem, jeśli siła dośrodkowa działająca na samochód jadący po łuku drogi jest zbyt mała to samochód wpadnie w poślizg.

\[
T = f \cdot N
\]

gdzie:
\begin{itemize}
  \item $f$ – współczynnik tarcia,
  \item $N$ – siła nacisku na podłoże.
\end{itemize}
\end{frame}

\begin{frame}{Przyczepność a opony}
\begin{itemize}
  \item Współczynnik tarcia nie jest stały.
  \item Zależy m.in. od:
    \begin{itemize}
      \item temperatury,
      \item pogody,
      \item rodzaju nawierzchni.
    \end{itemize}
  \item Opony wyścigowe w F1 mogą osiągać wartości
        nawet $f \approx 1{,}7$ na suchej nawierzchni.
\end{itemize}
\centering
  \includegraphics[
    width=0.6\paperwidth,
    height=0.5\paperheight,
    frame,
    margin=1pt,
    keepaspectratio,
  ]{images/obrazek4.jpg}
\end{frame}

%-------------------------------------------------
\section{Docisk aerodynamiczny}

\begin{frame}{Docisk aerodynamiczny}
Oprócz tarcia mechanicznego, na przyczepność wpływa również
\textbf{siła aerodynamiczna}. Siła ta generowana jest tak samo, jak siła nośna działająca w samolocie. 
\begin{block}{Kluczowa idea}
Działa ona analogicznie do siły nośnej w samolocie,
lecz jest skierowana \textbf{w dół}, dociskając bolid do toru.
\end{block}
Skrzydło w bolidzie kieruje powietrze w górę, a na podstawie zasady akcji-reakcji powietrze odpycha samochód w dół, zwiększając jego ciężar poprzez docisk aerodynamiczny.
\end{frame}



\begin{frame}{Elementy generujące docisk}
Największą rolę w generowaniu docisku aerodynamicznego odgrywają:
\begin{enumerate}
  \item Tylne skrzydło,
  \item Dyfuzor (podłoga),
  \item Przednie skrzydło.
\end{enumerate}
\end{frame}

%-------------------------------------------------
\section{Skrzydła aerodynamiczne}

\begin{frame}{Tylne skrzydło – zasada działania}
Zadaniem tylnego skrzydła jest nakierować powietrze tak aby jak najefektywniej opływało resztę auto. Nachylenie płatów skrzydeł ma kluczową rolę w znalezieniu kompromisu pomiędzy dociskiem a prędkością. Czyli spoiler:
\begin{itemize}
  \item Kieruje powietrze nad i za bolidem.
  \item Kąt nachylenia skrzydła decyduje o kompromisie:
    \begin{itemize}
      \item większy docisk vs. większy opór,
      \item większa prędkość maksymalna vs. stabilność.
    \end{itemize}
\end{itemize}
\end{frame}

\begin{frame}{Dyfuzor - zasada działania}
Znajdujący się na spodzie bolidu, generuje siłę docisku (przyczepność), kierując powietrze przepływające pod podłogą.
\begin{itemize}
  \item Jego zadaniem jest:
    \begin{itemize}
      \item Utrzymanie przepływu bez turbulencji lub nadmiernego oporu.
      \item Tworzenie obszaru niskiego ciśnienia.
    \end{itemize}
  \item Stanowi prawie połowę całkowitej siły docisku bolidu, jego kształt i wymiar jest ściśle regulowany przez przepisy. 
\end{itemize}
\end{frame}

\begin{frame}{Przednie skrzydło i nos bolidu}
W przednie skrzydle płaty odginają się do tyłu tak samo jak w tylnym chodzi tu przede wszystkim o zmniejszenie oporu aerodynamicznego na prostej.
\begin{itemize}
  \item Przednie skrzydło pracuje najefektywniej blisko nawierzchni.
  \item Tworzy się wąski kanał przepływu powietrza.
  \item Zwiększenie prędkości przepływu spadek ciśnienia.
  \item Powstaje efekt przyssania przedniej osi bolidu do toru.
\end{itemize}


\end{frame}

%-------------------------------------------------
\section{Wpływ docisku}

\begin{frame}{Skala sił aerodynamicznych}
\begin{itemize}
  \item Nowoczesne bolidy F1 są zdolne do generowania przeciążenia rzędu \textbf{3,5 G}.
  \item Przy prędkości ok. \textbf{183 km/h} docisk aerodynamiczny jest równy ciężarowi bolidu.
\end{itemize}

\begin{block}{Ciekawostka}
Teoretycznie bolid mógłby jechać po suficie.
\end{block}
\end{frame}

%-------------------------------------------------
\section{Zakończenie}

\begin{frame}{Podsumowanie}
\begin{itemize}
  \item Aerodynamika jest kluczowa dla osiągów w F1.
  \item Docisk zwiększa przyczepność bez zwiększania masy.
  \item Odpowiedni balans między dociskiem a oporem
        decyduje o konkurencyjności bolidu.
\end{itemize}
\end{frame}

\begin{frame}{}
    \centering
    {\Huge \textbf{Dziękuję za uwagę}}\\[1em] 
    Bartłomiej Posłuszny
\end{frame}


%-------------------------------------------------
\section{Bibliografia}

\begin{frame}{Bibliografia}
\scriptsize
	\large {Źródła}
\begin{itemize}
  \item wikipedia.com
  \item formula1.com
  \item parcfer.me
  \item cyrkf1.pl
  \item Google grafika
\end{itemize}
	\large {Grafika}
\begin{itemize}
 \item https://assets.astonmartinf1.com/public
 \item https://blogger.googleusercontent.com
 \item https://www.cyrkf1.pl/wp-content/uploads
\end{itemize}
\end{frame}

\end{document}
